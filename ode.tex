\sep
\section{ODE}
\sep

\subsection{Introduction}
\Def ODE: An equation for an unknown function $f$
\begin{enumerate}
\item[•] f is a function of one variable
\item[•] The equation relates $f(x)$ to the values of its derivatives at the same point
\item[•] Order of ODE: order of the highest derivative
\end{enumerate}
\sep

\subsection{Linear ODEs}
\Def Linear ODE is an equation of the form
\[ y^{(k)} + a_{k - 1} y^{(k - 1)} + \cdots + a_0 y = b, \text{ where} \] 
$y = f(x)$ is the unknown function \\
$a_{k - 1}(x), ..., a_0(x), b(x)$ are continuous functions \\

\Def Linear homogeneous ODE: $ \quad b(x) = 0$ \\
\Def Linear inhomogeneous ODE: $ \ b(x) \neq 0$ \\

\Def Initial Value Problem for ODE: Specifying values of $y, y', ... , y^{(k - 1)}$ at an initial point $x_0$
\[y(x_0) = y_0, y'(x_0) = y_1, ..., y^{(k - 1)}(x_0) = y_{k - 1} \]

\sep

\Theorem[2.2.3] $I \subset \R$,  linear ODE of order $k \geq 1$
\begin{enumerate}
\item[(1)] Let $S_0$ be the set of solutions for $b = 0$. Then is $S_0$ a vector space of dimension k.
\item[(2)] For any initial conditions, there is a unique solution $f \in S_0$, s.t.
\[y(x_0) = y_0, y'(x_0) = y_1, ..., y^{(k - 1)}(x_0) = y_{k - 1} \]

\item[(3)] For an arbitrary b, the set of solutions is $S_b = \{f + f_p | f \in S_0\}$, where $f_p$ is a particular solution
\item[(4)] For any initial value problem, there is a unique solution $f \in S_b$
\end{enumerate}
\Bem If $b \neq 0$, then $S_b$ is not a vector space \\
\Bem If $f_1, f_2$ are solutions for $b_1(x), b_2(x)$, \\ $f_1 + f_2$ is a solution for $b_1(x) + b_2(x)$
\sep

\subsection{Linear ODEs of order 1}
\Def Consider linear ODE of order 1: $y' + ay = b $
\begin{enumerate}
\item[1.] Solve homogeneous equation $y' + ay = 0$ 
\item[2.] Find a solution of inhomogeneous equation, s.t. $S_b$ contains $f_h + f$ where $f \in S$.
\end{enumerate}
\Bem The solutions are given by $f_h + z f_1$, where $z \in \C$ and $f_1$ is a basis of $S$ \\
\Bem To solve the real value problem $f(x_0) = y_0$, one can solve $f_h(x_0) + z f_1(x_0) = y_0$ \\
\Bem If $a \in \R$, then there exists $f_h, f_1 \in \R$ \\

\sep

\Procedure Solving homogeneous equations
\[ f_h(x) = z \cdot e^{-A(x)} \text{ for } z \in \C\]

\sep

\Procedure Solving inhomogeneous equations
\begin{align*}
f_p(x) &= \int b(x) \cdot e^{A(x)} dx \cdot e^{-A(x)} \\
f(x) &= f_h(x) + f_p(x) \\
&= z \cdot e^{-A(x)} + \int b(x) \cdot e^{A(x)} dx \cdot e^{-A(x)}
\end{align*}
\sep

\subsection{Linear ODE with constant coefs. }
The equation takes the form: Let $a_{k - 1}, ..., a_0 \in \C$
\[ y^{(k)} + a_{k - 1} y^{(k - 1)} + ... + a_0 y = b(x) \] 
\sep

\Procedure Solving homogeneous equations \\
We look for solutions of the form $f(x) = e^{\alpha x}, \alpha \in \C$
\begin{align*}
0 &=y^{(k)} + a_{k - 1} y^{(k - 1)} + ... + a_0 y \\
&=  e^{\alpha x}( \alpha^k + a_{k - 1} \alpha^{k - 1} + \cdots + a_1 \alpha + a_0) \\
&= e^{\alpha x} P(\alpha)
\end{align*}

\Theorem $f$ is a solution if and only if $P(\alpha) = 0$. \\
\Bem According to the Fundamental Theorem of Algebra, there are k roots for $P$ in $\C$. \\
\Bem $P(\alpha)$ is the \textbf{characteristic polynomial} and the roots are called \textbf{eigenvalues}\\

\sep

\textbf{Case 1:} k distinct solutions for $P(\alpha) = 0$ \\
$f_j(x) = e^{\alpha_j x}$ are linearly independent. \\

Every solution for the ODE is of the form:
\[ f(x) = z_1 e^{\alpha_1 x} + \cdots + z_k e^{\alpha_k x}, \text{ with }  z_1, ..., z_2 \in \C\]

\sep

\textbf{Case 2:} $\exists \alpha$, which is a root of order $2 \leq j \leq k$
\[ f_{\alpha, 0}(x) = x^0 e^{\alpha x}, \cdots, f_{\alpha, j - 1}(x) = x^{j - 1}  e^{\alpha x}\]
Taking the union of the functions $f_{\alpha, j}$ for all roots of P, each with its multiplicity, gives a basis of the space of solutions. 

\sep

\Procedure Solving inhomogeneous equations: \\
Find a solution of inhomogeneous equation, s.t. $S_b$ contains $f_h + f$ where $f \in S$.

\sep
\Procedure Variation of Constants \\
Let $(f_1, f_2, \cdots , f_k)$ be a basis for the $f_h$ \\
$f_p = z_1(x) f_1(x) + \cdots + z_k(x) f_k(x)$, where
\[
\left\{\begin{array}{l}
z_{1}^{\prime}(x) f_{1}(x)+\cdots+z_{k}^{\prime}(x) f_{k}(x)=0 \\
z_{1}^{\prime}(x) f_{1}^{\prime}(x)+\cdots+z_{k}^{\prime}(x) f_{k}^{\prime}(x)=0 \\
\cdots \\
z_{1}^{\prime}(x) f_{1}^{(k-2)}(x)+\cdots+z_{k}^{\prime}(x) f_{k}^{(k-2)}(x)=0
\end{array}\right.
\]

\sep

\Trick Guess the particular solution

\begin{tabular}{ |p{1.8cm}||p{3.85cm}|  }
\hline
$b(x)$ & $f_p(x)$\\
\hline\hline
$a e^{\alpha x}$ & $k e^{\alpha x}$ \\
\hline
$a\sin(\beta x)$ & $k_1  \sin(\beta x) + k_2 \cos(\beta x)$ \\
$a\cos(\beta x)$ & \\
\hline
$a e^{\alpha x} \sin(\beta x)$ & $ e^{\alpha x}  [ k_1 \sin(\beta x) + k_2 \cos(\beta x)]$ \\
$a e^{\alpha x} \cos(\beta x)$ & \\
\hline
\end{tabular}
\Bem works also for $a = a(x)$, then $k = k(x)$