\section{Das Riemann Integral}

\subsection{Integrabilitätskriterien}

\Def[5.1.1] Eine Partition ist eine endliche Teilmenge $P \subset [a, b]$, wobei $a,b \in P$ \\

\Def[Untersumme]
\[s(f, P):=\sum_{i=1}^{n} f_{i} \delta_{i}, \quad \quad f_i = \inf_{x_{i-1}\leq x \leq x_i} f(x)\]
\[s(f):=\sup_{P \in \mathcal{P}(I)} s(f, P) \] 

\Def[Obersumme]
\[S(f, P):=\sum_{i=1}^{n} F_{i} \delta_{i}, \quad \quad F_i = \sup_{x_{i-1}\leq x \leq x_i} f(x)\]
\[S(f):=\inf_{P \in \mathcal{P}(I)} S(f, P) \] 


\Lemma[5.1.2] Sei $P'$ Verfeinerung von $P$
\[s(f, P) \leq s(f, P') \leq S(f, P') \leq S(f, P) \] 


\Def [5.1.3] Eine beschränkte Funktion $f : [a,b] \rightarrow \R$ ist integrierbar falls
\[s(f) = S(f) \quad := \int_{a}^{b} f(x) dx \] 

\Satz [5.1.4] Eine beschränkte Funktion $f : [a,b] \rightarrow \R$ ist integrierbar falls
\[\forall \varepsilon>0 \quad \exists P \in \mathcal{P}(I) \quad \text { mit } \quad S(f, P)-s(f, P)<\varepsilon\]

\Satz[5.1.8] Eine beschränkte Funktion $f: [a, b] \rightarrow \R$ ist genau dann integrierbar, falls $\forall \epsilon > 0 \ \exists \delta > 0$
\[ \forall P \in P_\delta (I), \ S(f, P) - s(f, P) < \epsilon \]

\sep 

\subsection{Integrierbare Funktionen}

\Satz[5.2.1] $f, g : [a,b] \rightarrow \R$ beschränkt, integrierbar
Dann sind $f + g$, $\lambda \cdot f$, $f \cdot g$, $\abs{f}$, $min(f, g)$, $ \frac{f}{g}$ (falls $g(x) > 0)$ integrierbar \\

\Korollar[5.2.3] $P, Q$ Polynom und $Q(x) \neq 0 \ \forall x \in [a,b]$
\[ \frac{P(x)}{Q(x)} \text{ ist integrierbar} \]

\Satz[5.2.4] $f: D \rightarrow \R$ ist \textbf{gleichmässig stetig}, falls $ \forall \epsilon > 0 \ \ \exists \delta > 0 \ \forall x, y \in D$
\[  \abs{x - y} < \delta \implies \abs{f(x) - f(y)} < \epsilon \]

\Satz[5.2.6] Falls $f : [a, b] \rightarrow $ stetig in $[a, b]$ ist. \\
Dann ist f in $[a,b]$ gleichmässig stetig. \\

\Satz[5.2.7] $f$ stetig $\implies f$ integrierbar \\

\Satz[5.2.8] $f$ monoton $\implies f$ integrierbar

\Satz[5.2.10] $f_1, f_2 : [a,b] \rightarrow$ integrierbar
\[\int_{a}^{b} (f_1(x) + f_2(x)) dx = \int_{a}^{b} f_1(x) dx + \int_{a}^{b} f_2(x) dx \]
\[\int_{a}^{b} (\lambda f(x)) dx = \lambda \int_{a}^{b} f(x) dx \]

\sep

\subsection{Ungleich. \& Mittelwertsatz}

\Satz[5.3.1] $f, g : [a,b] \rightarrow \R$ beschränkt, integrierbar \\
Sei \(f(x) \leq g(x) \ \ \forall x \in [a,b] \), dann folgt
\[ \int_{a}^{b} f(x) dx \leq \int_{a}^{b} g(x) dx \]

\Korollar[5.3.2] $f : [a, b] \rightarrow \R$ beschränkt, integrierbar
\[ \abs{\int_{a}^{b} f(x) dx} \leq \int_{a}^{b} \abs{f(x)} dx \]

\Satz[5.3.3] $f, g : [a,b] \rightarrow \R$ beschränkt, integrierbar
\[ \abs{ \int_a^b f(x) g(x) dx } \leq \sqrt{\int_a^b f^2(x) dx} \sqrt{\int_a^b g^2(x) dx} \] 

\Satz[5.3.4] $f : [a.b] \rightarrow \R$ stetig
\[ \exists \xi \in [a, b] \quad \text{mit} \quad \int_a^b f(x) dx = f(\xi) (b-a) \]

\Satz[5.3.6] $ f: [a, b] \rightarrow \R$, wobei $f$ stetig, $g$ beschränkt, integrierbar mit $g(x) \geq 0 \quad \forall x \in [a,b]$
\[ \exists \xi \in [a, b] \quad \text{mit} \quad \int_a^b f(x) g(x) dx = f(\xi) \int_a^b g(x) dx  \]
\sep

\subsection{Fundamentalsatz}

\Satz[5.4.1] $a < b$ und $f : [a,b] \rightarrow \R$ stetig
\[ F(x) = \int_a^x f(t) dt, \quad a \leq x \leq b \]
$F(x)$ ist stetig differenzierbar und $F'(x) = f(x)$

\Def[5.4.2] $F(x)$ ist die Stammfunktion von $f$ \\

\Satz[5.4.3 Fundamentalsatz] $f : [a,b] \rightarrow \R$ stetig
\[ \int_a^b f(x) dx = F(b) - F(a) \]

\sep

\Satz[5.4.5 Partielle Integration] ${f, g : [a,b] \rightarrow \R}$ stetig differenzierbar und $a < b$.
\[ \int_{a}^{b} f(x) g'(x) dx = \Big[f(x)g(x)\Big]_a^b - \int_{a}^{b} f'(x) g(x) dx \]

\Satz[5.4.6 Substitution] $ \phi : [a,b] \rightarrow \R$ stetig differenzierbar, $\phi([a,b]) \subset I $, $f : I \rightarrow \R$ stetig.
\[ \int_{\phi(a)}^{\phi(b)} f(x) dx=\int_{a}^{b} f(\phi(t)) \phi^{\prime}(t) dt \]

\Korollar[5.4.8] $f : I \rightarrow \R$ stetig, $a,b,c \in \R$ 
\[ \int_{a + c}^{b + c} f(x) dx = \int_{a}^{b} f(t + c) dt \]
\[ \int_{a}^{b} f(ct) dt = \frac{1}{c} \int_{ac}^{bc} f(x) dx \]

\sep

\subsection{Integration konv. Reihen}

\Satz[5.5.1] Sei ${f_n : [a,b] \rightarrow \R}$ eine Folge von beschränkten, integrierbaren Funktionen, die gleichmässig gegen $f: [a,b] \rightarrow \R$ konvergiert. Dann ist f beschränkt und integrierbar
\[ \lim\limits_{n \rightarrow \infty} \int_a^b f_n(x) dx = \int_a^b f(x) dx \]

\Korollar[5.5.2] Sei $f_n$ ist eine Folge beschränkter integrierbarer Funktion, so dass $\sum_{n=0}^\infty f_n$ gleichmässig konvergiert 

\[ \sum_{n=0}^\infty \int_a^b f_n(x) dx= \int_a^b \Bigg( \sum_{n=0}^\infty f_n(x) \Bigg ) dx \] %

\Korollar[5.5.3] Potenzreihe ist integrierbar $\forall x \in \ ]-p, p[$
\[\int_0^x \sum_{n= 0}^\infty c_n x^n =  \sum_{n= 0}^\infty \frac{c_n}{n + 1} x^{n+1} \]

\subsection{Uneigentliche Integrale}

\Def[5.8.1] Sei $f:[a,\infty] \rightarrow \R$ beschränkt und integrierbar auf $[a,b] \quad \forall b \geq a$, wir definieren
\[\int_a^\infty f(x) dx := \lim_{b \rightarrow \infty} \int_a^b f(x) dx\]
Grenzwert existiert $\implies$  $f$ auf $[a, + \infty]$ integrierbar \\

\Korollar[5.8.2] $f: [a, \infty[ \ \rightarrow \R$ beschränkt, integrierbar
\begin{enumerate}
\item Falls $\abs{f(x)} \leq g(x) \quad \forall x \geq$ und $g(x)$ ist auf $[a, +\infty[ \ $ integrierbar, so ist $f$ auf $[a, +\infty[ \ $ integrierbar
\item Falls $0 \leq g(x) \leq f(x)$ und $\int_a^\infty g(x) dx$ divergiert, so divergiert auch $\int_a^\infty f(x) dx$
\end{enumerate}

\Satz[5.8.5] Sei $f : [1, \infty[ \ \rightarrow [0, \infty [$ monoton fallend
\[ \sum_{n=1}^\infty \text{ konvergiert} \Longleftrightarrow \int_1^\infty f(x) dx \text{ konvergiert}  \]

\Def[5.8.8] Falls $f$ auf $[a+\epsilon,b], \epsilon>0$ beschränkt, integrierbar ist, aber nicht beschränkt auf $]a,b]$
\[ \int_a^b f(x) dx := \lim_{\epsilon \rightarrow 0} \int_{a+\epsilon}^b f(x) dx \]
Grenzwert existiert $\implies$  $f$ auf $[a, b]$ integrierbar \\

\sep

\Def[5.8.11 Gamma Funktion] Für $s > 0$
\[ \Gamma(x) := \int_0^\infty e^{-x} x^{s - 1} dx \]

\Satz[5.8.12]
\begin{enumerate}
\item Die Gamma Funktion erfüllt die Relationen
\begin{enumerate}
\item $\Gamma(1) = 1$
\item $\Gamma(s + 1) = s \Gamma(s) \quad\forall s > 0 $
\item $\Gamma$ ist logarithmisch konvex für $0 \leq \lambda \leq 1$
\[ \Gamma(\lambda x + (1 - \lambda) y) \leq \Gamma(x)^\lambda \lambda(y)^{1-\lambda} \ \  \forall x, y > 0\]
\item $\Gamma(n + 1) = n!$
\item $\Gamma(x)=\lim _{n \rightarrow+\infty} \frac{n ! n^{x}}{x(x+1) \cdots(x+n)} \ \ \forall x>0$

\end{enumerate}
\item Die Gamma Funktion ist die einzige Funktion $] 0 , \infty [ $, die (a), (b), (c) erfüllt
\end{enumerate}

\sep

\subsection{Das unbestimmte Integral}

\Satz[5.1.9] $R(x) = \frac{P(x)}{Q(x)}$ eine rationale Funktion und $\text{grad}(P) < \text{grad}(Q)$, dann ist
\begin{align*}
Q(x) &= x^n + a_{n-1}x'{n-1}+\dots \\
	 &=	\prod_{i=1}^{k}\left(x-\gamma_{i}\right)^{n_{i}} \prod_{j=1}^{l}\left(\left(x-\alpha_{j}\right)^{2}+\beta_{j}^{2}\right)^{m_{j}} 
\end{align*}
und
\begin{align*}
\frac{P(x)}{Q(x)}=&\sum_{i=1}^{k} \sum_{j=1}^{n_{i}} \frac{C_{i j}}{\left(x-\gamma_{i}\right)^{j}}+ \\
				 &\sum_{i=1}^{l} \sum_{j=1}^{m_{i}} \frac{\left(A_{i j}+B_{i j} x\right)}{\left(\left(x-\alpha_{i}\right)^{2}+\beta_{i}^{2}\right)^{j}}
\end{align*}

\sep