\section{Stetige Funktionen}

\subsection{Reelwertige Funktionen}

\Def[3.1.1] Sei $f \in \R^D$
\begin{enumerate}
\item[(1)] f ist \textbf{nach oben beschränkt}, falls $f(D) \subset \R$ nach oben beschränkt ist.
\item[(2)] f ist \textbf{nach unten beschränkt}, falls $f(D) \subset \R$ nach unten ebschränkt ist.
\item[(3)] f ist \textbf{beschränkt}, falls $f(D) \subset \R$ beschränkt ist.
\end{enumerate}

\Def[3.1.2] Eine Funktion $f: D \rightarrow \R$ ist 
\begin{enumerate}
\item[(1)] \textbf{monoton wachsend}, falls $\forall x, y \in D$
\[
x \leqslant y \Longrightarrow f(x) \leqslant f(y)
\]
\item[(2)] \textbf{streng monoton wachsend}, falls $\forall x, y \in D$
\[
x<y \Longrightarrow f(x)<f(y)
\]
\item[(3)] \textbf{monoton fallend}, falls $\forall x, y \in D$
\[
x \leqslant y \Longrightarrow f(x) \geqslant f(y)
\]
\item[(4)] \textbf{streng monoton fallend}, falls $\forall x, y \in D$
\[
x<y \Longrightarrow f(x)>f(y)
\]
\item[(5)] \textbf{monoton}, falls $f$ monoton wachsend oder monoton fallend
\item[(6)] \textbf{streng monoton}, falls $f$ streng monoton wachsend oder streng monoton fallend ist.
\end{enumerate}

\sep

\subsection{Stetigkeit an einem Punkt}
\Def[3.2.1] Sei $x_{0} \in D \subseteq \R$. Die Funktion $f: D \rightarrow \R$ ist in $\boldsymbol{x}_{0}$ stetig, falls $\forall \epsilon > 0 \ \exists \delta > 0 \ \forall x \in D$
\[ \abs{x-x_{0}}<\delta \Longrightarrow\abs{f(x)-f\left(x_{0}\right)}<\varepsilon\]  

\Def[3.2.2] Die Funktion $f: D \rightarrow \R$ ist genau dann stetig, falls sie in jedem Punkt von $D$ stetig ist.  \\

\Satz[3.2.4] Die Funktion $f : D \rightarrow \R$ ist genau dann in x0 stetig, falls für jede Folge $a_n$
\[ \lim \limits_{n \rightarrow \infty} a_{n}=x_{0} \Longrightarrow \lim \limits_{n \rightarrow \infty} f\left(a_{n}\right)=f\left(x_{0}\right) \] 

\Korollar[3.2.5] Seien ${f : D \rightarrow \R} \text{ und } {g : D \rightarrow \R}$ beides Funktionen, welche in $x_0$ sind, dann gilt
\begin{enumerate}
\item[1)] $f + g, \ \lambda \cdot f, \ f \cdot g$ stetig in $x_0$
\item[2)] falls $g(x_0) \neq 0, \frac{f}{g}$ stetig in $x_0$
\end{enumerate}

\sep

\Def[3.2.6] Polynomiale Funktion $P : \R \rightarrow \R$: 
\[P(x) = a_n x^n + \cdots + a_0 \]
wobei $a_n, \cdots, a_0 \in \R$ und Grad ist n, falls $a_n \neq 0$

\Korollar[2.3.7] $P(x)$ ist auf ganz $\R$ stetig

\sep

\subsection{Zwischenwertsatz} 

\Satz[3.3.1] $I \subset \R, $ $f: I \rightarrow \R$ stetig und $a, b \in I$ \\
Für jedes c zwischen $f(a)$ und $f(b)$ gibt es ein z zwischen a und b mit $f(z) = c$.

\Korollar[3.3.2] Ein Polynom n-ten grades mit n ungerade, hat mindestens eine Nullstelle in n.
\sep

\subsection{Min-Max Satz}

\Def[3.4.2] $D \subset \R, f : D \rightarrow \R$, $g  : D \rightarrow \R$
\begin{itemize}
\item[•] $D = [a, b], a \leq b$ ist in dieser Form kompakt
\item[•] $max(f, g) (x) := max (f(x), g(x)) \quad  \forall x \in D$
\item[•] $min(f, g) (x) := min (f(x), g(x)) \quad  \forall x \in D$
\item[•]$\abs{f}(x) := \abs{f(x)}$
\end{itemize} 

\Lemma[3.4.3] Sei $x_0 \in D$ und f, g stetig in $x_0$. \\
Dann sind $\abs{f}, \ max(f, g), min(f, g)$ stetig in $x_0$ \\

\Lemma[3.4.4] $\{x_n : n \geq 1\} \subset [a, b] \Rightarrow  \lim\limits_{n \veryshortarrow \infty} x_n \in [a, b]$

\Satz[3.4.5] $f: I =  [a, b] \rightarrow \R$ stetig, dann $\exists u, v \in I$
\[ \text{so dass } f(u) \leq f(x) \leq(v) \ \ \forall x \in I \text{(f ist beschränkt)} \]

\sep

\subsection{Umkehrabbildungen}

\Satz[3.5.1] $D_1, D_2 \subset \R, \ x_0 \in D_1, f : D_1 \rightarrow D_2$ in $x_0$ stetig, $f(x_0) \in D_2, \ g : D_2 \rightarrow \R$ in $f(x_0)$ stetig
\[ \implies g \circ f : D_1 \rightarrow \R  \text{ stetig in } x_0 \] 

\Satz[3.5.3]$f \rightarrow \R$ stetig, streng monoton, dann ist $J := f(I) \subset \R$ ein Intervall und $f^{-1} : J \rightarrow I$ stetig, streng monoton wachsend

\sep

\subsection{Exponentialfunktion}
\[ \exp(x) = \sum_{n=0}^\infty \frac{x^n}{n!} \]
\Satz[3.6.1] exp : $\R \rightarrow ]0, + \infty[$ ist streng monoton wachsend, stetig und surjektiv. \\

\Korollar[3.6.2] exp(x) $> 0 \quad \forall x \in \R$ \\

\Bsp[2.7.27] $\forall z, w  \exp(w + z) = \exp(w) + \exp(z)$\\

\Korollar[3.6.3] exp(z) $>$ exp(y) $\forall z > y$\\

\Korollar[3.6.4] exp(x) $\geq  1 + x \quad \forall x \in \R$ \\
 $ \implies \exp(-x) \leq \frac{1}{1+x} \quad \forall x \in \R$ \\

\Korollar[3.6.5] $ \ln : ]0, + \infty[ \ \rightarrow \R$ ist streng monoton wachsend, stetig und bijektiv. \\
Es gilt \( \ln(a \cdot b) = \ln a + \ln b \quad \forall a, b \in \ ]0, +\infty[ \)

\Korollar[3.6.6] $f : \ ]0, + \infty[ \ \rightarrow \  ]0, + \infty[$
\begin{enumerate}
\item Für $a > 0$ ist $f(x) = x^a$ eine stetige, streng monoton wachsende Bijektion
\item Für $a < 0$ ist $f(x) = x^a$ eine stetige, streng monoton fallende Bijektion
\item $\ln(x^a) = a \ln(x) \quad \forall a \in \R, \ \forall x > 0$
\item $x^a \cdot x^b = x^{a + b} \quad \forall a, b \in \R, \ \forall x > 0$
\item $(x^a)^b = x^{a \cdot b} \quad \forall a, b \in \R, \ \forall x > 0$
\end{enumerate}

\sep

\subsection{Konvergenz v. Funktionenfolgen}

\Def[3.7.1] $f_n$ \textbf{konvergiert punktweise} gegen ${f : D \rightarrow \R}$, falls für alle $x \in D$ :
\[f(x) =  \lim\limits_{n \rightarrow \infty} f_n(x) \]
 
\Def[3.7.3] $f_n : D \rightarrow \R$ \textbf{konvergiert gleichmässig} in D gegen $f : D \rightarrow R$, falls $ \forall \epsilon \geq 0, \ \exists N > 1$, so dass
\[\forall n \geq N, \ \ \forall x \in D : \abs{f_n(x) - f(x)} < \epsilon \]
\[\limsup \limits_{x \in I} \abs{f(x) - f_n(x) } \xrightarrow[n \rightarrow \infty]{} 0 \]

\Satz[3.7.4] Falls $f_n : D \rightarrow \R$ gegen $f : D \rightarrow \R$ gleichmässig konvergiert, dann ist f stetig. \\

\Satz[3.7.5] $f_n : D \rightarrow \R$ ist \textbf{gleichmässig konvergent}, falls für alle x $f(x) = \lim\limits_{n \rightarrow \infty} f_n(x)$ existiert
und $f_n$ gleichmässig gegen f konvergiert. \\

\Korollar[3.7.6] $f_n : D \rightarrow \R$ konvergiert genau dann gleichmässig in D, falls $ \forall \epsilon \geq 0, \ \exists N > 1$, so dass
\[\forall n, m \geq N \text{ und } \forall x \in D \abs{f_n(x) - f_m(x)} < \epsilon \]

\sep 

\Def[3.7.8] Eine Reihe $\sum_{k=0}^{\infty} f_k (x)$ \textbf{konvergiert gleichmässig}, falls die durch ${S_n = \sum_{k=0}^{n} f_k (x)}$ gegebene Funktionenfolge gleichmässig konvergiert \\

\Def[3.7.9] $f_n :  D \rightarrow \R$ eine Folge stetiger Funktionen, wobei $\abs{f_n (x)} \leq c_n \quad \forall x \in D$ und $\sum_{n=0}^\infty c_n$ konvergiert. \\ Dann konvergiert die Reihe $f(x) := \sum_{n=0}^\infty f_n(x)$ , wobei $f(x)$ eine stetige Funktion ist. 

\sep