\section{Stetige Funktionen}

\subsection{Reelwertige Funktionen}

\Def[3.1.1] Sei $f \in \R^D$
\begin{enumerate}
\item[(1)] f ist \textbf{nach oben beschränkt}, falls $f(D) \subset \R$ nach oben beschränkt ist.
\item[(2)] f ist \textbf{nach unten beschränkt}, falls $f(D) \subset \R$ nach unten ebschränkt ist.
\item[(3)] f ist \textbf{beschränkt}, falls $f(D) \subset \R$ beschränkt ist.
\end{enumerate}

\Def[3.1.2] Eine Funktion $f: D \longrightarrow \mathbf{R},$ wobei $D \subset \mathbf{R},$ ist
\begin{enumerate}
\item[(1)] \textbf{monoton wachsend}, falls $\forall x, y \in D$
\[
x \leqslant y \Longrightarrow f(x) \leqslant f(y)
\]
\item[(2)] \textbf{streng monoton wachsend}, falls $\forall x, y \in D$
\[
x<y \Longrightarrow f(x)<f(y)
\]
\item[(3)] \textbf{monoton fallend}, falls $\forall x, y \in D$
\[
x \leqslant y \Longrightarrow f(x) \geqslant f(y)
\]
\item[(4)] \textbf{streng monoton fallend}, falls $\forall x, y \in D$
\[
x<y \Longrightarrow f(x)>f(y)
\]
\item[(5)] \textbf{monoton}, falls $f$ monoton wachsend oder monoton fallend
\item[(6)] \textbf{streng monoton}, falls $f$ streng monoton wachsend oder streng monoton fallend ist.
\end{enumerate}

\subsection{Stetigkeit an einem Punkt}
\Def[3.2.1] Sei $x_{0} \in D \subseteq \R$. Die Funktion $f: D \longrightarrow \R$ ist in $\boldsymbol{x}_{0}$ stetig, falls es für jedes $\varepsilon>0$ ein $\delta>0$ gibt, so dass für alle $x \in D$ gilt: $$\abs{x-x_{0}}<\delta \Longrightarrow\abs{f(x)-f\left(x_{0}\right)}<\varepsilon$$.
\sep
\sep
\sep
% Allgemeine Stetigkeit
\Def Die Funktion $f: D \longrightarrow \R$ ist stetig falls sie in jedem Punkt von $D$ stetig ist.

\section{Stetigkeit an einem Punkt}

% Epsilon-Delta Definition
\Def[Epsilon] Sei $x_{0} \in D \subseteq \R$. Die Funktion $f: D \longrightarrow \R$ ist in $\boldsymbol{x}_{0}$ stetig, falls es für jedes $\varepsilon>0$ ein $\delta>0$ gibt, so dass für alle $x \in D$ gilt: $$\abs{x-x_{0}}<\delta \Longrightarrow\abs{f(x)-f\left(x_{0}\right)}<\varepsilon$$.

% Sequence Definition
\Satz[Sequence] Sei $x_{0} \in D \subseteq \R$. Die Funktion $f: D \longrightarrow \R$ ist genau dann in $\boldsymbol{x}_{0}$ stetig, falls für jede Folge $\left(a_{n}\right)_{n \geqslant 1}$ in $D$
	$$\lim \limits_{n \rightarrow \infty} a_{n}=x_{0} \Longrightarrow \lim \limits_{n \rightarrow \infty} f\left(a_{n}\right)=f\left(x_{0}\right)$$ gilt.
	
% Limit
\Satz[Sidewise] Sei $x_{0} \in D \subseteq \R$. Die Funktion $f: D \longrightarrow \R$ ist in $\boldsymbol{x}_{0}$ stetig, falls 
	$$f(x_0) =\lim \limits_{x \rightarrow x_0} f(x) =  \lim \limits_{x \rightarrow x_0^+} f(x) = \lim \limits_{x \rightarrow x_0^-} f(x)$$ gilt.	


% Punktweise Konvergenz

% Gleichmässige Konvergenz