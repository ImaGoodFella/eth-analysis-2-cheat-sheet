\sep

\section{Differenzierbare Funktionen}


\Def[4.1.1] $f$ ist \textbf{${ \text{ in } x_0 \text{ differenzierbar}}$}, falls

\[ f'(x_0) = \lim\limits_{x \rightarrow x_0} \frac{f(x) - f(x_0)}{x - x_0} \text{ existiert} \]

\Bem $f'(x_0) = \lim\limits_{h \rightarrow 0} \frac{f(x_0 + h) - f(x_0)}{h}$

\sep
\subsection{Die Ableitung}

\Satz[4.1.3] $f: D \rightarrow \R$, $x_0 \in D$ ein Häufungspunkt
\begin{enumerate}
\item $f$ ist in $x_0$ differenzierbar
\item Es git $c \in \R$ und $r : D \rightarrow \R$ mit \\
$f(x) = f(x_0) + c(x - x_0) + r(x) \cdot (x - x_0)$ \\
$r(x_0) = 0$ und r ist stetig in $x_0$
\end{enumerate}
Dann ist $c = f'(x_0)$ eindeutig bestimmt. \\

\Satz[4.1.4] $f : D \rightarrow \R$ ist in $x_0$ differenzierbar, genau dann, wenn $\phi : D \rightarrow \R$ stetig in $x_0$ ist und
\[ f(x) = f(x_0) + \phi(x) \cdot (x - x_0), \quad \text{wobei } \phi(x_0) = f'(x_0) \]

\Korollar[4.1.5] $f$ differenzierbar in $x_0 \Rightarrow$ $f$ stetig in $x_0$ \\

\Def[4.1.7] $f: D \rightarrow \R$ ist in $D$ differenzierbar, falls $\forall x_0 \in D$, $f$ in $x_0$ differenzierbar ist  \\

\Satz[4.1.9] Sei $f, g : D \rightarrow  \R \text{ in } x_0$ differenzierbar

\sep

\begin{enumerate}
\item $f + g$ ist in $x_0$ differenzierbar und
\[ (f + g)'(x_0) = f'(x_0) + g'(x_0) \]
\item $f \cdot g$ ist in $x_0$ differenzierbar und
\[ (f \cdot g) (x_0) = f'(x_0) \cdot g(x_0) + f(x_0) \cdot g'(x_0) \]
\item Falls $g(x_0) \neq 0$ ist $ \frac{f}{g}$ in $x_0$ differenzierbar
\[ \bigg(\frac{f}{g} \bigg)' (x_0) = \frac{f'(x_0) \cdot g(x_0) - f(x_0) \cdot g'(x_0)}{g(x_0)^2} \]
\end{enumerate}

\Satz[4.1.11] $f: D \rightarrow E$ in $x_0$ und $g: E \rightarrow \R$ in $f(x_0)$ differenzierbar, so ist $(g \circ f$) in $x_0$ differenzierbar
\[ (g \circ f)' (x_0) = g' (f(x_0)) \cdot f'(x_0)  \]

\Korollar[4.1.12] $f : D \rightarrow E$ bijektiv
\[ \implies f^{-1} \text{ differenzierbar, } (f^{-1})' (y_0) = \frac{1}{f'(x_0)}  \]

\sep

\subsection{Erste Ableitung}

\Def[4.2.1] $f: D \rightarrow \R$ und $x_0 \in D$
\begin{enumerate}
\item $f$ hat lokales Maximum in $x_0$, falls $\delta > 0$ gibt
\[ f(x) \leq f(x_0) \quad \forall x \in \ ]x_o - \delta, x_0 + \delta[ \ \cap  \ D\]
\item $f$ hat lokales Minimum in $x_0$, falls $\delta > 0$ gibt
\[ f(x) \geq f(x_0) \quad \forall x \in  \ ]x_o - \delta, x_0 + \delta[ \ \cap  \ D\]
\item Maximums und Minimums sind Extremums
\end{enumerate}

\Satz[4.2.2] $f : \ ]a,b[ \ \rightarrow \R$, $ f \text{ in } x_0$ differenzierbar
\begin{enumerate}
\item Falls $f'(x) > 0$ gibt es $\delta > 0$ \\
$f(x) > f(x_0) \quad \forall x \in \ ]x_0, x_0 + \delta [$ \\
$f(x) < f(x_0) \quad \forall x \in \ ]x_0 - \delta, x_0 [$

\item Falls $f'(x) < 0$ gibt es $\delta > 0$ \\
$f(x) < f(x_0) \quad \forall x \in \ ]x_0, x_0 + \delta [$ \\
$f(x) > f(x_0) \quad \forall x \in \ ]x_0 - \delta, x_0 [$

\item $x_0$ ist lokales Extremum $\implies f'(x) = 0$
\end{enumerate}

\sep 

\Satz[4.2.3] $f : [a,b] \rightarrow \R$ stetig, in $ ]a, b[ $ differenzierbar, so gibt es $\xi \in \ ]a,b[$ mit
\[ f'(\xi) = 0\]

\Satz[4.2.5] $f : [a,b] \rightarrow \R$ stetig, in $ ]a, b[ $ differenzierbar, so gibt es $\xi \in \ ]a,b[$ mit
\[f(b) - f(a) = f'(\xi) (b - a) \]

\Korollar[4.2.5] $f, g :[a,b] \rightarrow \R$ stetig, in $]a, b[$ differenzierbar und $\forall \xi \in \ ]a, b[$
\begin{enumerate}
\item $f'(\xi) = 0 $, $\Rightarrow$ $f$ konstant 
\item $f'(\xi) \geq 0$, $\Rightarrow$ $f$ monoton wachsend
\item $f'(\xi) > 0$, $\Rightarrow$ $f$ streng monoton wachsend
\item $f'(\xi) \leq 0$, $\Rightarrow$ $f$ monoton fallend
\item $f'(\xi) < 0$, $\Rightarrow$ $f$ streng monoton fallend
\item $f'(\xi) = g'(\xi)$, $\Rightarrow$ $ \exists c \in \R$ mit $f(x) = g(x) + c$
\item $\abs{f'(\xi)} \leq M$ $\Rightarrow$ $\abs{f(x_1) - f(x_2)} \leq M \abs{x_1 - x_2}$
\end{enumerate}

\sep

\Satz[4.2.9] $f, g : [a,b] \rightarrow \R$ stetig, differenzierbar in $]a, b[$, falls $g'(x) \neq 0 \quad \forall x \in \ ]a, b[$  folgt 
\[g(a) \neq g(b) \text{ und } \frac{f(b) - f(a}{g(b) - g(a)} = \frac{f'(\xi)}{g'(\xi)}\]

\Satz[4.2.10 l'Hospital] $f, g : [a, b] \rightarrow \R$ differenzierbar mit $g'(x) \neq 0 \quad \forall x \in \ ]a, b[$ 
\[ \lim\limits_{x \rightarrow b^-} f(x) =  \lim\limits_{x \rightarrow b^-} g(x) = 0 \ \ \land \lim\limits_{x \rightarrow b^-} \frac{f'(x)}{g'(x)} = \lambda \] 
\[ \implies \lim\limits_{x \rightarrow b^-} \frac{f(x)}{g(x)} = \lim\limits_{x \rightarrow b^-} \frac{f'(x)}{g'(x)}  = \lambda \]

\Bem Gilt auch für $b = +\infty, \lambda = + \infty, x \rightarrow a^+$

\sep

\Def[4.2.13] $f : I \rightarrow \R$, $x \leq y$, bzw. $x < y$
\begin{enumerate}
\item $f$ ist \textbf{konvex} falls $\forall x,y \in \land \ \forall \lambda \in [0,1] $
\[ f(\lambda \cdot x + (1 - \lambda) \cdot y) \leq \lambda \cdot f(x) + (1 - \lambda) \cdot f(y) \]

\item $f$ ist \textbf{streng konvex} falls $\forall x,y \in \land \ \forall \lambda \in  \ ]0,1[ $
\[ f(\lambda \cdot x + (1 - \lambda) \cdot y) < \lambda \cdot f(x) + (1 - \lambda) \cdot f(y) \]
\end{enumerate}

\Bem $f(x)$ ist \textbf{konkav}, falls $-f(x)$ konvex ist \\

\Lemma[4.2.15] f ist konvex $\Leftrightarrow$ für alle $x_0 < x < x_1 \in I$
\[ \frac{f(x) - f(x_0)}{x - x_0} \leq \frac{f(x_1) - f(x)}{x_1 - x} \]

\Bem $f$ streng konvex $\Leftrightarrow$ strikte Ungleichung gilt \\

\Satz[4.2.16] Sei $f : \ ]a, b[ \rightarrow \R$ differenzierbar
\[ f \text{ (streng) konvex} \Leftrightarrow f'(x)  \text{ (streng) monoton wachs.} \]

\Korollar[4.2.17] $f : \ ]a, b[ \rightarrow \R$ differenzierbar
\[ f \text{ konvex} \Leftrightarrow f''(x)  \geq 0 \]
\[ f \text{ streng konvex} \Leftrightarrow f''(x)  > 0 \]

\sep

\subsection{Höhere Ableitungen}

\Def[4.3.1] Sei $f : D \rightarrow \R$ differenzierbar
\begin{enumerate}
\item $f$ ist \textbf{n-mal differenzierbar}, falls $f^{(n -1)}$ in D differenzierbar ist. $f^{(n)} := (f^{(n-1)})'$
\item $f$ ist \textbf{n-mal stetig differenzierbar}, falls f n-mal differenzierbar und $f^{(n)}$ stetig ist
\item $f$ ist \textbf{glatt}, falls $\forall n \geq 1$ $f$ n-mal differenzierbar
\end{enumerate}

\Satz[4.3.3] $f, g : D \rightarrow \R$ n-mal differenzierbar
\begin{enumerate}
\item $f + g$ ist n-mal differenzierbar
\[ (f + g)^{(n)} = f^{(n)} + g^{(n)} \] 
\item $f \cdot g$ ist n-mal differenzierbar
\[ (f \cdot g)^{(n)} = \sum_{k=0}^n \binom{n}{k} f^{(k)} g^{(n - k)} \]
\end{enumerate}

\Satz[4.3.5] $f, g : D \rightarrow \R$ n-mal differenzierbar
\[ \text{Falls } g(x) \neq 0 \ \ \forall x \in D, \frac{f}{g} \text {ist n-mal differenzierbar} \]

\Satz[4.3.6] $f,g : D \rightarrow \R$ n-mal differenzierbar 
\[(g \circ f)^{(n)} (x) = \sum_{k=1}^n A_{n,k}(x) (g^{(k)} \circ f) (x) \]

\Bem $A_{n, k}$ ist Polynom in $f', f^{(2)}, \cdots , f^{(n + 1 - k)}$

\sep

\subsection{Potenzreihen \& Taylor Approx.}

\Satz[4.4.1] $f_n : \ ]a, b[ \ \rightarrow \R$, wobei $f_n$ einmal stetig differenzierbar ist, $f_n$ und $f'_n$ gleichmässig konvergieren. Dann ist $f$ stetig differenzierbar.
\[ \lim\limits_{n \rightarrow \infty} f_n = f \text{ und } \lim\limits_{n \rightarrow \infty} f'_n = f' \]

\Satz[4.4.2] $ \sum_{k= 0}^\infty c_k x^k$ eine Potenzreihe
\[ f(x) = \sum_{k= 0}^\infty c_k x^k \text{ ist differenzierbar} \]
\[f'(x) = \sum_{k= 0}^\infty k c_k x^k\]

\Korollar[4.4.3] $f(x) = \sum_{k= 0}^\infty c_k x^k$ ist glatt 

\sep

\Satz[4.4.5] $f : [a,b] \rightarrow \R \ (n + 1)$-mal differenzierbar \\
Für jedes $a < x \leq b$ gibt es $\xi \in \ ]a,x[ \ $
\[ f(x) = \sum_{k=0}^n \frac{f^{(k)}(a)}{k!} (x-a)^k  + \frac{f^{(n + 1)}(\xi)}{(n + 1)!} (x - a)^{n + 1} \]

\Korollar[4.4.6] $f : [c,d] \rightarrow \R \ (n + 1)$-mal differenzierbar \\
Sei $c < a < d$, so folgt für alle $x \in [c,d]$ $x \leq \xi \leq a$
\[ f(x) = \sum_{k=0}^n \frac{f^{(k)}(a)}{k!} (x-a)^k  + \frac{f^{(n + 1)}(\xi)}{(n + 1)!} (x - a)^{n + 1} \]

\sep

\Korollar[4.4.7] $n \geq 0, a < x_0 < b$ und $f : D \rightarrow \R \ (n + 1)$-mal stetig differenzierbar \\
Annahme: $f'(x_0) = f^{(2)}(x_0) = \text{...} = f^{(n)} (x_0) = 0$
\begin{enumerate}
\item Falls n gerade und $x_0$ lokale Extremalstelle folgt $f^{(n + 1)}(x_0) = 0$
\item Falls n ungerade und $f^{(n + 1)}(x_0)  > 0$, so ist $x_0$ eine strikte lokale Minimalstelle
\item Falls n ungerade und $f^{(n + 1)}(x_0)  < 0$, so ist $x_0$ eine strikte lokale Maximalstelle
\end{enumerate}

\Korollar[4.4.8] Sei $f : D \rightarrow \R$ zweimal stetig differenzierbar. Sei $ a < x_0 < b $. Annahme $f'(x_0) = 0$.
\begin{enumerate}
\item Falls $f^{(2)}(x_0) > 0$ so folgt daraus, dass $x_0$ strikte lokale Minimalstelle ist.
\item Falls $f^{(2)}(x_0) < 0$ so folgt daraus, dass $x_0$ strikte lokale Maximalstelle ist.
\item Falls $f^{(2)}(x_0) = 0$ und $f^{(3)}(x_0) \neq 0$, so ist $x_0$ ein Sattelpunkt
\end{enumerate}

\Bem Falls $f^{(2)}(x_0) = 0$ und $f^{(3)}(x_0) \neq 0$, so ist $x_0$ ein Wendepunkt (hier ist $f'(x)$ beliebig)