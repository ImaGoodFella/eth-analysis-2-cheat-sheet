\section{Folgen und Reihen}

\Def[2.1.1] Eine \textbf{Folge} $a_n$ ist eine Abbildung
\[
a: \mathbf{N}^{*} \longrightarrow \mathbf{R}
\] 
\subsection{Konvergenz von Folgen}

\Def[2.1.4] Eine Folge $a_n$ heisst \textbf{konvergent}, falls es $a \in \R$ gibt, so dass $\forall \epsilon > 0$ die Menge
$
\left\{n \in \mathbf{N}^{*}: a_{n} \notin\right] a-\varepsilon, a+\varepsilon[\}
$ endlich ist.
\Lemma[2.1.3] Dieses $a$ ist \textbf{eindeutig}. \\

\Lemma[2.1.5] Jede konvergente Folge ist \textbf{beschränkt}. \\
\Achtung $a_{n}$ beschränkt $\centernot\implies a_{n}$ konvergent! \\

\Lemma[2.1.6] Eine Folge $a_n$ \textbf{konvergiert} gegen ${a = \lim\limits_{n \rightarrow \infty} a_{n}}$, falls $ \forall \epsilon > 0 \ \exists N \geq 1$ so dass $\ \forall n\geq N$
\[
 \abs{a_n-a} < \epsilon.
\]

\sep

\Satz[2.1.8] Seien $a_n$ und $b_n$ konvergente Folgen mit $a = \lim\limits_{n \rightarrow \infty} a_{n}$ und  $b = \lim\limits_{n \rightarrow \infty} b_{n}$
\begin{enumerate}
\item[1)] Dann ist $\lim\limits_{n \rightarrow \infty} (a_{n} + b_{n}) = a + b$
\item[2)] Dann ist $\lim\limits_{n \rightarrow \infty} (a_{n} \cdot b_{n}) = a \cdot b$
\item[3)] Dann ist $\lim\limits_{n \rightarrow \infty} (\frac{a_{n}}{b_{n}} = \frac{a}{b})$ $(b_{n} \neq 0 \ \forall n \geq 1)$
\item[4)] $\exists K \geq 1 \ \forall n \geq K \ a_{n} \leq b_{n} \implies a \leq b$

\end{enumerate}

\subsection{Der Satz von Weierstrass}

% Monotone Konvergenz
\Satz[Monotone] Sei $\left(a_{n}\right)_{n \geqslant 1}$ monoton fallend und nach unten beschränkt. Dann konvergiert
$\left(a_{n}\right)_{n \geqslant 1}$ mit Grenzwert
$
\lim _{n \rightarrow \infty} a_{n}=\inf \left\{a_{n}: n \geqslant 1\right\}
$.

% Cauchy
\Satz[Cauchy] Die Folge $\left(a_{n}\right)_{n \geqslant 1}$ ist genau dann konvergent, falls $\forall \varepsilon>0 \quad \exists N \geqslant 1 \quad$ so dass $\quad\left|a_{n}-a_{m}\right|<\varepsilon \quad \forall n, m \geqslant N$

% Sandwich
\Satz[Sandwich] Die Folge $\left(a_{n}\right)_{n \geqslant 1}$ konvergiert zu $a$, falls $\left(b_{n}\right)_{n \geqslant 1}, \left(c_{n}\right)_{n \geqslant 1}$ existieren mit Grenzwert $a$ und $\forall n\geq1 : b_n \leq a_n \leq c_n$.


\subsection{Konvergenz von Reihen}

% Absolute Convergence
\Def Die Reihe $\sum_{k=1}^{\infty} a_{k}$ konvergiert absolut ($\Rightarrow$ konvergent), falls $\sum\limits_{k=1}^{\infty} \abs{a_{k}}$ kovergiert.

% Cauchy
\Satz[Cauchy] Die Reihe $\sum_{k=1}^{\infty} a_{k}$ ist genau dann konvergent, falls. $\forall \varepsilon>0 \quad \exists N \geqslant 1 \quad$ mit $\quad\left|\sum\limits_{k=n}^{m} a_{k}\right|<\varepsilon \quad \forall m \geqslant n \geqslant N$

% Quotientenkriterum
\Satz[Ratio] Sei $\left(a_{n}\right)_{n \geqslant 1}$ mit $a_{n} \neq 0 \quad \forall n \geqslant 1 .$ Falls 
$$\limsup\limits_{n \rightarrow \infty} \frac{\left|a_{n+1}\right|}{\left|a_{n}\right|}<1$$ dann konvergiert die Reihe absolut.
Falls $\liminf\limits_{n \rightarrow \infty}\square > 1$ divergiert die Reihe.

% Wurzelkriterum
\Satz[Root] Falls $$\limsup\limits_{n \rightarrow \infty} \sqrt[n]{\left|a_{n}\right|}<1$$ dann konvergiert $\sum_{n=1}^{\infty} a_{n}$ absolut. Falls $\square > 1$, dann divergiert die Reihe.

% Alternierende Reihe
\Satz[Alternating] Sei $\left(a_{n}\right)_{n \geqslant 1}$ monoton fallend mit $a_{n} \geqslant 0 \quad \forall n \geqslant 1$ und $\lim \limits_{n \rightarrow \infty} a_{n}=0 .$ Dann konvergiert 
$$S:=\sum_{k=1}^{\infty}(-1)^{k+1} a_{k}$$ und es gilt $a_{1}-a_{2} \leqslant S \leqslant a_{1}$.


% % % % %
% TRICKS
% % % % %
\subsection{Andere Aussagen}
\Lemma (Bernouilli) $(1+x)^{n} \geqslant 1+n \cdot x \quad \forall n \in \N, x>-1$.

% Bolzano Weierstrass
\Satz[Teilfolge] Jede beschränkte Folge besitzt eine konvergente Teilfolge.

% R^d
\Satz[Vektorfolge] $\lim \limits_{n \rightarrow \infty} a_{n}=b$ genau dann wenn $\lim \limits_{n \rightarrow \infty} a_{n, j}=b_{j} \quad \forall 1 \leqslant j \leqslant d$.