\sep
\section{Folgen und Reihen}

\Def[2.1.1] Eine \textbf{Folge} $a_n$ in $\R$ ist eine Abbildung
\[
a: \N\longrightarrow \R
\] 
\sep
\subsection{Konvergenz von Folgen}

\Def[2.1.4] Eine Folge $a_n$ heisst \textbf{konvergent}, falls es $a \in \R$ gibt, so dass $\forall \epsilon > 0$ die Menge
$
\left\{n \in \N: a_{n} \notin\right] a-\varepsilon, a+\varepsilon[\}
$ endlich ist.
\Lemma[2.1.3] Dieses $a$ ist \textbf{eindeutig}. \\

\Lemma[2.1.5] Jede konvergente Folge ist \textbf{beschränkt}. \\
\Achtung $a_{n}$ beschränkt $\centernot\implies a_{n}$ konvergent! \\

\Lemma[2.1.6] Eine Folge $a_n$ \textbf{konvergiert} gegen ${a = \lim\limits_{n \rightarrow \infty} a_{n}}$, falls $ \forall \epsilon > 0 \ \exists N \geq 0$ so dass $\ \forall n\geq N$
\[
 \abs{a_n-a} < \epsilon.
\]

\Satz[2.1.8] Seien $a_n$ und $b_n$ konvergente Folgen mit $a = \lim\limits_{n \rightarrow \infty} a_{n}$ und  $b = \lim\limits_{n \rightarrow \infty} b_{n}$
\begin{enumerate}
\item[1)] Dann ist $\lim\limits_{n \rightarrow \infty} (a_{n} + b_{n}) = a + b$
\item[2)] Dann ist $\lim\limits_{n \rightarrow \infty} (a_{n} \cdot b_{n}) = a \cdot b$
\item[3)] Dann ist $\lim\limits_{n \rightarrow \infty} (\frac{a_{n}}{b_{n}}) = \frac{a}{b}$ $(b_{n} \neq 0 \ \forall n \geq 0)$
\item[4)] $\exists K \geq 0 \ \forall n \geq K \ a_{n} \leq b_{n} \implies a \leq b$

\end{enumerate}

\Satz[Sandwich Satz] Sei $\lim\limits_{n \rightarrow \infty} a_{n} = \lim\limits_{n \rightarrow \infty} b_{n} = \alpha$ 
\[
a_{n} \leq c_{n} \leq b_{n} \ \forall n \geq K \implies \lim\limits_{n \rightarrow \infty} c_{n} = \alpha
\]

Die Folge $a_{n}$ ist divergent, falls sie nicht konvergiert.
\sep

\subsection{Weierstrass und Anwendungen}

\Def[2.2.1] Falls  $a_{n}$ ist
\begin{enumerate}
\item[1)] \textbf{monoton wachsend}  falls ${a_{n} \leq a_{n + 1} \ \forall n \geq 0}$
\item[2)] \textbf{monoton fallend} falls ${a_{n} \geq a_{n + 1} \ \forall n \geq 0}$
\end{enumerate}

\Satz[2.2.2 (Weierstrass)] Genau dann, wenn $a_{n}$ 
\begin{enumerate}
\item[1)] \textit{monoton wachsend} und \textit{nach oben beschränkt} ist, dann \textbf{konvergiert} $a_{n}$ mit Grenzwert
\[ 
\lim\limits_{n \rightarrow \infty} a_{n} = \text{sup}\{a_{n} \: n \geq 0\}
\]
\item[2)] \textit{monoton fallend} und \textit{nach unten beschränkt} ist, dann \textbf{konvergiert} $a_{n}$ mit Grenzwert
\[ 
\lim\limits_{n \rightarrow \infty} a_{n} = \text{inf}\{a_{n} \: n \geq 0\}
\]
\end{enumerate}
\Bsp[2.2.3] $\lim\limits_{n \rightarrow \infty} n^{a}q^{n} = 0, \ 0 \leq q < 1, \ a \in \Z$ \\
\Bsp[2.2.5] $\lim\limits_{n \rightarrow \infty} \sqrt[n]{n} = 1$ \\
\Lemma[2.2.7 (Bernoulli Ungleichung)]
\[
(1 + x)^{n+1} \geq 1 + n \cdot x \ \ \forall n \in \N, x > -1
\] 

\sep

\subsection{Limes superior und inferior}

\Def[2.3.0]
\[
\liminf\limits_{n \rightarrow \infty} a_{n} = \lim\limits_{n \rightarrow \infty} b_{n}, \ \ (b_{n} = \text{inf} \{a_{k} : k \geq n \})
\]
\[
\limsup\limits_{n \rightarrow \infty} a_{n} = \lim\limits_{n \rightarrow \infty} c_{n}, \ \ (c_{n} = \text{sup} \{a_{k} : k \geq n \})
\]

\Lemma[2.4.1] Die Folge $a_{n}$ konvergiert genau dann, falls
\begin{enumerate}
\item[1.] $a_{n}$ beschränkt ist
\item[2.] $\limsup\limits_{n \rightarrow \infty} a_{n} = \liminf\limits_{n \rightarrow \infty} a_{n}$
\end{enumerate}

\sep 

\subsection{Cauchy Kriterium}

\Satz[2.4.2 (Cauchy Kriterium)] Die Folge $a_{n}$ ist genau dann konvergent und heisst Cauchy-Folge
\[
\forall \epsilon > 0 \ \exists N \geq 0 \text{ so dass } \abs{a_{n} - a_{m}} \ \ \forall n, m \geq N
\]

\sep

\subsection{Bolzano-Weierstrass}

\Def[2.5.1] Ein abgeschlossenes Intervall $I \subset \R$ 
\begin{enumerate}
\item[1)] $\left[ a, b \right]$, $a \leq b, a, b \in \R$
\item[2)] $\left[ a, +\infty \right[$, $a \in \R$
\item[3)] $\left] -\infty, a \right]$, $a \in \R$
\item[4)] $\left] -\infty, +\infty \right] = \R$
\end{enumerate}

Die Länge eines $\mathcal{L}(I)$ ist definiert als:
\begin{enumerate} 
\item[•] $\mathcal{L}(I) = b - a$   \quad im ersten Fall
\item[•] $\mathcal{L}(I) = \infty$  \qquad in (2), (3), (4)
\end{enumerate}
\sep

\Satz[2.5.5 (Cauchy-Cantor)]
\\ Sei ${I_{1} \subseteq I_{2} \subseteq \cdots I_{n} \subseteq \cdots}$ eine Folge abgeschlossener Intervall mit $\mathcal{L}(I_{1}) < +\infty$. Dann gilt
\[ {\bigcap}_{n\geq1} I_{n} \neq \emptyset \]
\[ \lim\limits_{n \rightarrow \infty} \mathcal{L}(I_{n}) = 0 \implies \abs{{\bigcap}_{n\geq1} I_{n}} = 1\]

\Satz[2.5.6] $\R$ ist nicht \textbf{abzählbar}.

\sep

\Def[2.5.7] $b_{n}$ ist eine Teilfoge von $a_{n}$, falls 
\[b_{n} = a_{l(n)}, \quad l : \N\rightarrow \N\text{ und } l(n) > l(n + 1) \]

\Satz[2.5.9 (Bolzano-Weierstrass)] Für jede beschränkte Folge existiert eine konvergente Teilfolge.

\sep

\subsection{Folgen in $R^{d}$ und C}

\Def[2.6.1] Eine \textbf{Folge} $a_n$ in $\mathbf{R^{d}}$ ist eine Abbildung
\[
a: \N\longrightarrow \mathbf{R^{d}}
\] 

\Def[2.6.2] Eine Folge $a_n$ in $\mathbf{R^{d}}$ konvergiert gegen ${a = \lim\limits_{n \rightarrow \infty} a_{n}}$, falls $ \forall \epsilon > 0 \ \exists N \geq 0$ so dass $\ \forall n\geq N$
\[
 \norm{a_n-a} < \epsilon.
\]
\sep

\subsection{Reihen}

\Def[2.7.0] Eine Reihe ist eine unendliche Summe
\[S_{n} := a_{1} + a_{n} \cdots = \sum_{k=0}^{n} a_{k}\]

\Def[2.7.1] Die Reihe $\sum_{k=1}^{n} a_{k}$ ist \textbf{konvergent}, falls die Folge der Partialsummen konvergiert. 
\[\sum_{k=0}^{\infty} a_{k} = \lim\limits_{n \rightarrow \infty} S_{n} \]

\Bsp[2.7.2 (Geometrische Reihe)] Sei $\abs{q} < 1$
\[\sum_{k=0}^{\infty} q^k = \frac{1}{1-q} \]

\Satz[2.7.4] Seien $\sum_{k=0}^{\infty} a_{k}$ und $\sum_{k=0}^{\infty} b_{k}$ konvergent
\begin{enumerate}
\item[(1)] $\sum_{k=0}^{\infty} (a_{k} + b_{k}) = \sum_{k=0}^{\infty} a_{k} + \sum_{k=0}^{\infty} a_{k}$ 
\item[(2)] $\sum_{k=0}^{\infty} \alpha \cdot a_{k} = \alpha \cdot \sum_{k=0}^{\infty} a_{k}$
\end{enumerate}

\sep

\Satz[2.7.5 (Cauchy Kriterium)] $\sum_{k=0}^{\infty} a_{k}$ ist genau dann konvergent, falls
\[ \forall \epsilon > 0 \ \exists n \geq 0 \text{ mit } \abs{\sum_{k=n}^{m} a_{k}} < \epsilon \quad \forall m \geq n \geq N\]

\Satz[2.7.6] Sei $\sum_{k=0}^{\infty} a_{k}$ mit $a_{k} \geq  0 \ \forall k \in \N$. Dann konvergiert
$\sum_{k=0}^{\infty} a_{k}$ genau dann, falls die Folge $S_{n} = \sum_{k=0}^{n} a_{k}$ nach oben beschränkt ist

\sep

\Korollar[2.7.7 (Vergleichssatz)] Seien $\sum_{k=0}^{\infty} a_{k}$ und $\sum_{k=0}^{\infty} b_{k}$ Reihen mit: $0 \leq a_{k} \leq b_{k} \quad \forall k \geq 0. $
\[ \sum_{k=0}^{\infty} b_{k} \text{ konvergent} \implies \sum_{k=0}^{\infty} a_{k} \text{ konvergent} \]
\[ \sum_{k=0}^{\infty} a_{k} \text{ divergent} \implies \sum_{k=0}^{\infty} b_{k} \text{ divergent} \]

\sep

\Satz[2.7.9] $\sum_{k=0}^{\infty} a_{k}$ heisst \textbf{absolut konvergent}, 
\[ \text{falls } \sum_{k=0}^{\infty} \abs{a_{k}} \text{ konvergiert} \]

\Satz[2.7.10] Eine absolut konvergente Reihe $\sum_{k=0}^{\infty} a_{k}$ ist auch konvergent und es gilt:
\[ \abs{\sum_{k=0}^{\infty} a_{k}} \leq \sum_{k=0}^{\infty} \abs{a_{k}}\]

\sep

\Satz[2.7.12 Leibniz] Sei $a_{n}$ monoton fallend mit $a_{n} \geq 0 \ \forall n \geq 0, \ \lim\limits_{n \rightarrow \infty} a_{n} = 0.$ Dann konvergiert
\[ S :=  \sum_{k=0}^{\infty} (-1)^{k+1} a_{k} \text{ und } a_{1} - 1_{2} \leq S \leq {a_1} \]
\sep

\Def[2.7.14] Eine Reihe  $\sum_{k=0}^{\infty} a'_{n}$ ist eine Umordung der Reihe  $\sum_{k=0}^{\infty} a_{n}$, falls es eine bijektive Abbildung $\phi : \N^{*} \rightarrow \N^{*}$ mit $a'_{n} = a_{\phi(n)}$ 

\Satz[2.7.16 Dirichlet] Falls $\sum_{k=0}^{\infty} a_{n}$ absolut konvergiert, dann konvergiert jede Umordnung der Riehe und hat denselben Grenzwert.  

\Satz[Riemann] Sei $\sum_{k=0}^{\infty} a_{n}$ eine konvergente, aber nicht absolut konvergente Riehe, dann gibt es zu jedem $A \in \R \cup \{\pm \infty\}$ eine Umordnung der Reihe, die gegen A konvergiert. 

\sep

\Satz[Quotientenkriterium] Sei $a_{n} \neq 0 \ \forall n \geq 0$
\[\limsup\limits_{n \rightarrow \infty} \frac{\left|a_{n+1}\right|}{\left|a_{n}\right|}<1 \implies \sum_{n=0}^{\infty} a_{n} \ \text{konvergiert absolut}\]
\[\limsup\limits_{n \rightarrow \infty} \frac{\left|a_{n+1}\right|}{\left|a_{n}\right|}>1 \implies \sum_{n=0}^{\infty} a_{n} \ \text{divergiert} \quad \quad \quad \quad\]

\sep

\Satz[Wurzelkriterium] 
\[\limsup\limits_{n \rightarrow \infty} \sqrt[n]{\left|a_{n}\right|}<1 \implies \sum_{n=0}^{\infty} a_{n} \text{ konvergiert absolut} \]
\[\limsup\limits_{n \rightarrow \infty} \sqrt[n]{\left|a_{n}\right|}>1 \implies \sum_{n=0}^{\infty} a_{n} \text{ divergiert} \quad \quad \quad \quad \]

\Korollar[2.7.21] Die Potenzreihe $\sum_{k=0}^{\infty} c_{k}$ konvergiert für alle $\abs{z} < \rho$ und divergiert für alle $\abs{z} > \rho$
\[\rho =\begin{cases}
\quad \quad + \infty \quad \quad \text{ falls } \limsup\limits_{k \rightarrow \infty} \sqrt[k]{\left|c_{k}\right|} = 0 \\
\frac{1}{\limsup\limits_{c \rightarrow \infty} \sqrt[k]{\left|c_{k}\right|}}  \text{ falls } \limsup\limits_{k \rightarrow \infty} \sqrt[k]{\left|c_{k}\right|} > 0 \\
\end{cases}\]
\Bem Der Konvergenzbereich ist $\{z \in  \C | \abs{z} < \rho\}$

\sep

\Def[2.7.22] $\sum_{k=0}^{\infty} b_{k}$ ist eine lineare Anordnung der Doppelreihe $\sum_{i,j \geq 0} a_{ij}$, falls es eine Bijektion $\sigma : \N \rightarrow \N \times \N$ gibt mit $b_k = a_{\epsilon(k)}$.

\Satz[2.7.23] Falls $\sum_{i=0}^{m}\sum_{j=0}^{m}\abs{a_{ij}} \leq B, \quad \forall m \geq 0$
\[ \text{dann konvergiert } S_{i} := \sum_{j=0}^{\infty} a_{ij} \quad \forall i \geq 0 \]
\[ \text{dann konvergiert } U_{j} := \sum_{i=0}^{\infty} a_{ij} \quad \forall j \geq 0 \]
\[ \text{und es gilt } \sum_{i=0}^{m} S_{i} = \sum_{j=0}^{m} U_{j} \]

\Satz[2.7.24] Das \textbf{Cauchy Produkt} der Reihen $\sum_{i=0}^{\infty} a_i, \ \sum_{i=0}^{\infty} b_i$ ist die Reihe
\[\sum_{n=0}^\infty \Bigg(\sum_{j=0}^{n} a_{n-j} b_{j} \Bigg) = a_0 b_0 + (a_0 b_1 + a_1 b_0) + \cdots  \]

\Satz[2.7.26] Falls die Reihen $\sum_{i=0}^{\infty} a_i, \ \sum_{i=0}^{\infty} b_i$ absolut konvergieren, so knovergiert ihr Cauchy Produkt und es gilt:
\[\sum_{n=0}^\infty \Bigg(\sum_{j=0}^{n} a_{n-j} b_{j} \Bigg) = \Bigg( \sum_{i=0}^\infty a_i \Bigg) \Bigg(\sum_{j=0}^\infty b_j \Bigg) \]

\sep

\Satz[2.7.28] Sei $f_n : \N \rightarrow \R$ eine Folge, für die gilt:
\begin{enumerate}
\item[(1)] $ f(j) :=  \lim\limits_{n \rightarrow \infty}$ existiert $\forall j \in \N$ 
\item[(2)] Es gibt eine Funktion $g : \N \rightarrow [0, \infty[,$ so dass
\begin{enumerate}
\item[(2.1)] $\abs{f_n(j)} \leq g(j) \quad \forall j \geq 0, \ \forall n \geq 0$
\item[(2.2)] $\sum_{j=0}^{\infty} g(j)$ konvergiert
\end{enumerate}
\end{enumerate}

\( \text{Dann folgt } \sum_{j=0}^{\infty} f(j) =  \lim\limits_{n \rightarrow \infty} \sum_{j=0}^{\infty} f_n(j) \)

\Korollar[2.7.29] Für jedes $z \in \C$ gilt
\[\lim\limits_{n \rightarrow \infty} \sum_{j=0}^{\infty} \bigg(1 + \frac{z}{n} \bigg)^n = \text{exp}(z) \]
\sep